\documentclass[a4paper,12pt]{article}
\usepackage[margin=0.65in]{geometry}
\usepackage{amsmath,amssymb,amstext}
\usepackage[hidelinks]{hyperref}
\usepackage{graphicx}
\usepackage{float}

\usepackage{setspace}
\setlength{\parskip}{0.65cm}
\singlespacing

\setlength{\parindent}{0cm}

\begin{document}

\title{\texttt{OpenCCM} CSTR Validation: Reversible Linear Reaction}
\author{Matthew Peres Tino}
\maketitle

\section{Reaction System}

Suppose we have the following reversible linear reaction:
\begin{center}
	$A\xrightarrow[]{k_f} B$
	
	$B\xrightarrow[]{k_r} A$
\end{center}

The reaction rates are easily derived, and are as follows:
\begin{align}
    r_A &= k_rB - k_fA \\
    r_B &= k_fA - k_rB 
\end{align}
Herein, for a chemical species $P$, the concentration at a given time (i.e. $[P]$ or $C_P(t)$) will be noted as just $P$ for brevity.

\section{CSTR Mass Balance}

A transient mass balance for a species $P$ in a CSTR is as follows. 
\begin{align}
    \frac{dN_P}{dt} &= F_{in} - F_{out} + Vr_P\\
    V\frac{dP}{dt} &= Q P_{IN} - QP + Vr_P \\
    \frac{dP}{dt} &= \frac{P_{IN}}{\tau} - \frac{P}{\tau} + r_P
\end{align}
where $N_P$ is the number of moles, $F$ represents the molar flow rate in $mol/s$, $V$ is the volume of the CSTR, $r_P$ is the reaction rate, and $P$ and $P_{IN}$ represent the concentration of species $P$ and the inlet feed rate (which plays the role of a ``boundary condition''), respectively.
Solving for constant volume $V$ and volumetric feed rate $Q$ allows for simplifications as shown above, where $\tau = \frac{V}{Q}$.

\newpage 
\subsection{Mass Balance System}
Using the formula above, the mass balance for species $A$ is: 
\begin{equation}
    \frac{dA}{dt} = \alpha A + k_r B + \frac{A_{IN}}{\tau}
\end{equation}
where $\alpha = -k_f - 1/\tau$, and $A_{IN}$ represents the ``boundary condition'' (constant inlet feed rate) for species $A$. 

Similarly the mass balance for species $B$ is:
\begin{equation}
    \frac{dB}{dt} = k_f A + \beta B + \frac{B_{IN}}{\tau}
\end{equation}
where $\beta = -k_r - 1/\tau$, and $B_{IN}$ represents the ``boundary condition'' (constant inlet feed rate) for species $B$. 

This is a coupled linear system of inhomogeneous ODEs.
By recognizing that $r_A = -r_B$, this allows for a simplified total conservation of mass expression that is easy to solve analytically.

We begin by noting the following definitions:
\begin{align}
    C(t) &= A(t) + B(t)\\
    C_{IN} &= A_{IN} + B_{IN}\\
    C_0 &= A_0 + B_0
\end{align}
Now finding an expression for transient conservation of mass:
\begin{align}
    \frac{dC}{dt} &= \alpha A + k_r B + \frac{A_{IN}}{\tau} + k_f A + \beta B + \frac{B_{IN}}{\tau}\\
    &= (\alpha + k_f)A + (\beta + k_r)B + \frac{A_{IN} + B_{IN}}{\tau}\\
    &= \frac{-C}{\tau} + \frac{C_{IN}}{\tau}
\end{align}
This equation is ideal for use of the integrating factor method to generate a general solution for $C(t)$:
\begin{align}
    \frac{dC}{dt} + \frac{C}{\tau} &= \frac{C_{IN}}{\tau}\\
    \frac{d}{dt}(Ce^{t/\tau}) &=\frac{C_{IN}}{\tau} e^{t/\tau}\\
    Ce^{t/\tau} &= \int \frac{C_{IN}}{\tau} e^{t/\tau} dt\\
    Ce^{t/\tau} &= C_{IN}e^{t/\tau} + \gamma\\
    C(t) &= C_{IN} + \gamma  e^{-t/\tau}
\end{align}
Using a generalized initial condition $C(t=0) = C_0 = C_{IN} + \gamma$, we find that $\gamma = C_0 - C_{IN}$.
We now have $C(t)$ in its final form:
\begin{equation}
    C(t) = C_{IN} + (C_0 - C_{IN}) e^{-t/\tau}
\end{equation}
Returning to the governing ODE for species $A$ and inserting $B = C-A$, we have:
\begin{align}
    A' &= \alpha A + k_r (C-A) + \frac{A_{IN}}{\tau}\\
    &= (\alpha- k_r)A + k_r C + \frac{A_{IN}}{\tau}
\end{align}
We now define some simplifications,
\begin{align}
    \alpha^* &= k_r - \alpha = k_r + k_f + \frac{1}{\tau}\\
    f(t) &= k_r C_{IN} + k_r(C_0 - C_{IN}) e^{-t/\tau} + \frac{A_{IN}}{\tau}
\end{align}
of which now we can define our new governing ODE for species $A$:
\begin{equation}
    A' + \alpha^* A = f(t)
\end{equation}
Again, this suggests use of the integrating factor method to obtain the general solution for $A(t)$:
\begin{align}
    \frac{d}{dt}(Ae^{\alpha^* t}) &= f(t)e^{\alpha^* t}\\
    Ae^{\alpha^* t} &= \int f(t)e^{\alpha^* t} dt\\
    &= \int \left(k_r C_{IN} + \frac{A_{IN}}{\tau}\right)e^{\alpha^* t} + k_r(C_0 - C_{IN}) e^{t(\alpha^* - 1/\tau)} dt\\
    &= \frac{\left(k_r C_{IN} + A_{IN}/\tau\right)}{\alpha^*}e^{\alpha^* t} + 
        \frac{k_r(C_0 - C_{IN}) e^{t(\alpha^* - 1/\tau)}}{\alpha^* - 1/\tau} + \omega\\
    \rightarrow A(t) &= \frac{\left(k_r C_{IN} + A_{IN}/\tau\right)}{\alpha^*} + 
        \frac{k_r(C_0 - C_{IN}) e^{-t/\tau}}{\alpha^* - 1/\tau} + \omega e^{-\alpha^* t}
\end{align}
Inserting the general initial condition, $A(t=0) = A_0$, we solve for the constant $\omega$:
\begin{align}
    A(0) &= A_0\\
    A(0) &= \frac{\left(k_r C_{IN} + A_{IN}/\tau\right)}{\alpha^*} + \frac{k_r(C_0 - C_{IN})}{\alpha^* - 1/\tau} + \omega \\
    \rightarrow \omega &= A_0 - \frac{\left(k_r C_{IN} + A_{IN}/\tau\right)}{\alpha^*} - \frac{k_r(C_0 - C_{IN})}{\alpha^* - 1/\tau}
\end{align}
At last we arrive at the complete expression for $A(t)$:
\begin{equation}
    A(t) = A_0e^{-\alpha^* t} + \frac{\left(k_r C_{IN} + A_{IN}/\tau\right)}{\alpha^*}\left(1-e^{-\alpha^* t}\right) + 
    \frac{k_r(C_0 - C_{IN})}{\alpha^* - 1/\tau}\left(e^{-t/\tau}-e^{-\alpha^* t}\right)
\end{equation}

\newpage 

\section{Simulation Setup}

Using the reaction system previously described, the following describes initial conditions and inlet feed rates (``boundary conditions'') in units of [M] (concentration) that were employed for simulations. 
\begin{align}
	A_{0} &= 0\\
    B_{0} &= 0\\
	A_{IN} &= 1\\
    B_{IN} &= 0
\end{align}
This setup represents a CSTR with no initial chemical species present but includes a constant inlet feed rate (boundary condition) of 1M for species $A$ which will induce reaction behaviour as time evolves. 

The solution vectors $sol^a$ (analytical) and $sol^n$ (\texttt{OpenCCM}) use the same specific timesteps given by the numerical solver.
The error between these two solutions for a species computed at each timestep $i$ for $N$ total timesteps is a normalized absolute error:
\begin{equation}
	err = \frac{\sum_i^N \lvert sol^a_i - sol^n_i \rvert}{N}
\end{equation}


\subsection{Simulation for Small $k$}

Simulations were run using varying tolerances of 1e-3, 1e-7, 1e-10, and 1e-13 for both absolute and relative tolerances using the following $k$ values (units of $s^{-1}$):
\begin{align}
    k_f &= 1 \times 10^{-2}\\
    k_r &= 1 \times 10^{-4}
\end{align}
Note that these constants were chosen arbitrarily to represent a system with dominant $A$ concentration over $B$ production. 
Additionally, although stricter tolerances result in larger number of simulation timesteps, this is accounted for in the error equation (normalization by $N$).
\begin{figure}[H]
    \includegraphics[width=0.49\linewidth]{tol_1e-03_kr_1e-04_kf_1e-02}\hfill
    \includegraphics[width=0.49\linewidth]{tol_1e-07_kr_1e-04_kf_1e-02}
\end{figure}
\begin{figure}[H]
    \includegraphics[width=0.49\linewidth]{tol_1e-10_kr_1e-04_kf_1e-02}\hfill
    \includegraphics[width=0.49\linewidth]{tol_1e-13_kr_1e-04_kf_1e-02}
\end{figure}

\subsection{Simulation for Large $k$}

Simulations were run using varying tolerances of 1e-3, 1e-7, 1e-10, and 1e-13 for both absolute and relative tolerances using the following $k$ values (units of $s^{-1}$):
\begin{align}
    k_f &= 1\\
    k_r &= 1
\end{align}
Note that these constants were chosen arbitrarily to represent a system with near-equal steady state $A$ and $B$ concentrations. 
Additionally, although stricter tolerances result in larger number of simulation timesteps, this is accounted for in the error equation (normalization by $N$).
\begin{figure}[H]
    \includegraphics[width=0.49\linewidth]{tol_1e-03_kr_1e+00_kf_1e+00}\hfill
    \includegraphics[width=0.49\linewidth]{tol_1e-07_kr_1e+00_kf_1e+00}
\end{figure}
\begin{figure}[H]
    \includegraphics[width=0.49\linewidth]{tol_1e-10_kr_1e+00_kf_1e+00}\hfill
    \includegraphics[width=0.49\linewidth]{tol_1e-13_kr_1e+00_kf_1e+00}
\end{figure}

\section{Conclusions}

In conclusion:
\begin{itemize}
	\item \texttt{OpenCCM}'s CSTR implementation can handle linear reversible reactions, both for relatively small or large $k$ rate constants, as validated by comparison to an analytically derived mass balance system.
	\item Increasing the relative \& absolute simulation tolerances significantly reduces the overall error between analytical and numerical solutions.
\end{itemize}


\end{document}
